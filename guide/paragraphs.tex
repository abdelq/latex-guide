\section{Paragraphes}

Généralement, un paragraphe est séparé d'un autre par une ligne vide, comme ceci :

\lstinputlisting[firstline=4, lastline=6]{paragraphs/intro/breakline}

Il existe une autre façon de séparer ses paragraphes, grâce à la commande \lstinline|\par|.
On peut écrire les deux paragraphes précédents comme ceci :

\lstinputlisting[firstline=4, lastline=5]{paragraphs/intro/par}

Par défaut, les paragraphes ont une indentation de \emph{1.5 fois} la taille de la police.
De plus, il n'y a pas d'espace inséré entre les paragraphes.

\subsection{Alignement}

Par défaut, les paragraphes dans \LaTeX{} sont justifiés.
Pour modifier l'alignement, il existe trois environnements:

\begin{enumerate}
  \item \texttt{center}, qui centre le texte
  \item \texttt{flushleft}, qui aligne le texte à gauche
  \item \texttt{flushright}, qui aligne le texte à droite
\end{enumerate}

Par exemple :

\lstinputlisting[firstline=4, lastline=16]{paragraphs/alignment}

Les trois environnements mentionnés plus haut sont basés autour des commandes :

\begin{itemize}
  \item \lstinline|\raggedright|, équivalent à \texttt{flushleft}
  \item \lstinline|\raggedleft|, équivalent à \texttt{flushright}
  \item \lstinline|\centering|, équivalent à \texttt{center}
\end{itemize}

Ces commandes sont des commandes <<switch>>.
Elles changent l'alignement à partir du point où elles sont insérées, jusqu'à la fin du document,
sauf si une autre de ces commandes à été utilisée.

\subsection{Indentation}

Par défaut, \LaTeX{} n'indente pas le premier paragraphe d'une section.
Pour les paragraphes subséquants, la taille de l'indentation est déterminée par le paramètre \lstinline|\parindent|.
Par exemple :

\lstinputlisting[firstline=4, lastline=10]{paragraphs/indentation}

Il est possible de modifier la taille de l'indentation dans un paragraphe,
grâce à la commande \lstinline|\setlength|.

Dans l'exemple, \lstinline|\setlength{\parindent}{10pt}| fait en sorte que
le paragraphe en dessous est indenté de 10pt.

Quant à la commande \lstinline|\noindent|, elle permet de créer un paragraphe sans indentation.
Au contraire, pour indenter un paragraphe qui n'est pas indenté,
il est possible d'utiliser \lstinline|\indent|.
À noter que ça n'a aucun effet, lorsque \lstinline|\parindent| est à \emph{0}.
