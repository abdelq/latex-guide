\section{Listes}

Les listes sont créées comme ceci :

\lstinputlisting[firstline=6, lastline=9]{lists/intro}

\subsection{Listes non-ordonnées}

Une liste non-ordonnée est produite grâce à l'environnement \texttt{itemize}.
Chaque entrée de la liste doit être précédée par \lstinline|\item|.

\lstinputlisting[firstline=4, lastline=7]{lists/unordered}

Par défaut, chaque élément est indiqué par un rond noir.

\subsection{Listes ordonnées}

Les listes ordonnées ont une syntaxe similaire aux listes non-ordonnées.
Seul l'envrionnement change.

Les listes sont générées avec l'environnement \texttt{enumerate}.

\lstinputlisting[firstline=4, lastline=7]{lists/ordered}

Chaque élément de la liste se voit attribué un nombre selon la position dans la liste, en commençant par \emph{1}.

\subsection{Listes imbriquées}

Dans \LaTeX, il est possible d'insérer des listes dans d'autres listes, et ce peut importe le type.

\lstinputlisting[firstline=4, lastline=11]{lists/nested}

\subsection{Style de listes}

\subsubsection{Listes ordonnées}

Le style de numérotation change dépendemment de la profondeur dans la liste :

\lstinputlisting[firstline=8, lastline=23]{lists/style/ordered}

Par défaut, la numérotation se fait comme suit :

\begin{description}
  \item[Niveau 1] Chiffres arabes : \emph{1, 2, 3\dots}
  \item[Niveau 2] Lettres minuscules : \emph{a, b, c\dots}
  \item[Niveau 3] Chiffres romains minuscules : \emph{i, ii, iii\dots}
  \item[Niveau 4] Lettres majuscules : \emph{A, B, C\dots}
\end{description}

Cette façon de numéroter les listes peut être redéfinie comme ceci :

% \lstinputlisting{lists/style/reordered}
