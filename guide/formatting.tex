\section{Formatage}

\subsection{Gras, italique, souligné}

Il existe trois commandes de bases pour obtenir du gras, de l'italique et du souligné :
\lstinline|\textbf|, \lstinline|\textit| et \lstinline|\underline|.

\lstinputlisting[firstline=4, lastline=7]{formatting/bf-it-un}

Ces commandes peuvent être combinées, comme dans l'exemple plus haut.

\subsection{Mise en évidence du texte}

La commande \lstinline|\emph| permet de mettre l'accent sur une partie de texte.
Dans plusieurs cas, \lstinline|\emph| donne le même résultat que \lstinline|\textit|.

Toutefois, les deux commandes ne sont pas exactement les mêmes. \lstinline|\emph| dépend du contexte.
Dans un texte normal, la section où l'accent est mise est en italique.
C'est l'inverse si le texte est en italique, la section où l'accent est mise ne l'est pas.

\lstinputlisting[firstline=4, lastline=8]{formatting/emphasis}

Il est à noter que certains <<packages>> comme \emph{Beamer} modifient la commande \lstinline|\emph|.
